\setcounter{page}{2}
% !TEX root = ./report.tex
\section*{Цель работы}

Знакомство со средством дизассемблирования Sourcer, получение дизассемблированного кода ядра операционной системы Windows на примере обработчика прерывания INT 8h в virtual mode – специальном режиме защищенного режима (32-разрядный режим работы), который эмулирует реальный режим работы  вычислительной системы на базе процессоров Intel.

\section*{Задание}

Используя Sourcer получить дизассемблированный код обработчика аппаратного прерывания от системного таймера INT 8h.

На основе полученного кода составить алгоритм работы обработчика INT 8h.

\section*{Листинг кода}
 Далее будут представлены листинги прерывания int 8h и процедуры sub\_7

\subsection*{Листинг INT8h} 
\begin{lstlisting}[style={asm}]
;; Вызов сабрутины 7 (запрет прерываний)
020A:0746  E8 0070		call	sub_7			; (07B9)

;; Загрузка регистров на стек
020A:0749  06			push	es
020A:074A  1E			push	ds
020A:074B  50			push	ax
020A:074C  52			push	dx

;; ds <- 40h
020A:074D  B8 0040		mov	ax,40h
020A:0750  8E D8		mov	ds,ax

;; es <- 0
020A:0752  33 C0		xor	ax,ax			; Zero register
020A:0754  8E C0		mov	es,ax

;; обработка счетчиков
;; обращаемс к байтам rtc, делаем инкремент счетчика 6C
020A:0756  FF 06 006C		inc	word ptr ds:[6Ch]	; (0040:006C=0AC4Bh)
;; если не произошло переполнение первого счетчика (6C), goto loc_3
020A:075A  75 04				jnz	loc_3			; Jump if not zero
;; если переполнение все таки произошло, нужно сделать икремент второго счетчика (6E)
020A:075C  FF 06 006E		inc	word ptr ds:[6Eh]	; (0040:006E=12h)

020A:0760			loc_3:
;; Если второй счетчик не равен 24, переход в loc_4
020A:0760  83 3E 006E 18	cmp	word ptr ds:[6Eh],18h	; (0040:006E=12h)
020A:0765  75 15		jne	loc_4			; Jump if not equal
;; Если второй счетчик не равен 0B0h=176, переход в loc_4
020A:0767  81 3E 006C 00B0	cmp	word ptr ds:[6Ch],0B0h	; (0040:006C=0AC4Bh)
020A:076D  75 0D		jne	loc_4			; Jump if not equal
;; обнуляем оба счетчика
020A:076F  A3 006E		mov	word ptr ds:[6Eh],ax	; (0040:006E=12h)
020A:0772  A3 006C		mov	word ptr ds:[6Ch],ax	; (0040:006C=0AC4Bh)
;; присваиваем третьему счетчику значение 1 (прошел день)
020A:0775  C6 06 0070 01	mov	byte ptr ds:[70h],1	; (0040:0070=0)
;; ставим 4-ый бит ax в единицу
020A:077A  0C 08		or	al,8
020A:077C			loc_4:
;; записываем ax в стек
020A:077C  50			push	ax

;; обработка контроллера дисковода
;; производим декремент счетчика шага дисковода
020A:077D  FE 0E 0040		dec	byte ptr ds:[40h]	; (0040:0040=0A2h)
;; если счетчик не равен 0, переход в loc_5
020A:0781  75 0B		jnz	loc_5			; Jump if not zero

;; посылаем в порт дисковода сигнал о том, что требуется остановка
020A:0783  80 26 003F F0	and	byte ptr ds:[3Fh],0F0h	; (0040:003F=0)
020A:0788  B0 0C		mov	al,0Ch
020A:078A  BA 03F2		mov	dx,3F2h
020A:078D  EE			out	dx,al			; port 3F2h, dsk0 contrl output

020A:078E			loc_5:

;; чтение ax из стека
020A:078E  58			pop	ax
020A:078F  F7 06 0314 0004	test	word ptr ds:[314h],4	; (0040:0314=3200h)

;; Проверка 3 байта - флага четности PF
;; Если поднят второй бит, то вызов маскируемых прерываний разрешен
;; Происходит вызов пользовательского прерывания 1Ch с поощью int в loc_6
020A:0795  75 0C		jnz	loc_6			; Jump if not zero
;; lahf загружает статусные флаги в регистр AH: 
;; AH <- EFLAGS(SF:ZF:0:AF:0:PF:1:CF)
020A:0797  9F			lahf				; Load ah from flags
;; меняем местами ah, al
020A:0798  86 E0		xchg	ah,al
;; загружаем ax  на стек
020A:079A  50			push	ax
;; происходит вызов пользовательского прерывания 1Ch с помощью call 
;; и абсолютного адреса
020A:079B  26: FF 1E 0070	call	dword ptr es:[70h]	; (0000:0070=6ADh)
020A:07A0  EB 03		jmp	short loc_7		; (07A5)
020A:07A2  90			nop
020A:07A3			loc_6:
;; Вызов прерывания 1Ch
020A:07A3  CD 1C		int	1Ch			; Timer break (call each 18.2ms)
020A:07A5	loc_7:
;; Повторный вызов сабрутины sub_7
020A:07A5  E8 0011		call	sub_7			; (07B9)
;; В ah копируется ascii символ ' '
020A:07A8  B0 20		mov	al,20h			; ' '
;; и записывается в порт 20h
020A:07AA  E6 20		out	20h,al			; port 20h, 8259-1 int command
										;  al = 20h, end of interrupt
;; Восстановление значений регистров и возврат из функции
020A:07AC  5A			pop	dx
020A:07AD  58			pop	ax
020A:07AE  1F			pop	ds
020A:07AF  07			pop	es
;; Переход в адрес 07B0 - 164 = 064C
020A:07B0  E9 FE99		jmp	$-164h

020A:064C  1E				push	ds
020A:064D  50				push	ax
020A:06AA  58				pop	ax
020A:06AB  1F				pop	ds

020A:06AC  CF				iret	 ; Interrupt return

\end{lstlisting}

\subsection*{Листинг sub\_7} 
\begin{lstlisting}[style={asm}]
 
sub_7		proc	near
;; запись DS, AX в стек
020A:07B9  1E			push	ds
020A:07BA  50			push	ax
;; AX <- 40, начало области данных BIOS
020A:07BB  B8 0040		mov	ax,40h
020A:07BE  8E D8		mov	ds,ax
	
;; загрузка младшего байта регистра EFLAGS в AX
020A:07C0  9F			lahf				; Load ah from flags
;; Если DF == 0 и старший бит IOPL == 0
;; Сброс флага разрешения IF
;; Иначе щапрет прерываний (маскируемых) с cli
020A:07C1  F7 06 0314 2400	test	word ptr ds:[314h],2400h	; (0040:0314=3200h)
020A:07C7  75 0C		jnz	loc_9			; Jump if not zero
	
;; Префикс выдачи сигнала LOCK# (префикс блокировки шины) заставляет процессор 
;; установить сигнал LOCK# во время выполнения следующей за ним команды. 
;; *Сброс флага IF*
020A:07C9  F0> 81 26 0314 FDFF          lock	and	word ptr ds:[314h],0FDFFh	; (0040:0314=3200h)
020A:07D0	loc_8:
020A:07D0  9E			sahf				; Store ah into flags
020A:07D1  58			pop	ax
020A:07D2  1F			pop	ds
020A:07D3  EB 03		jmp	short loc_10		; (07D8)
020A:07D5	loc_9:
020A:07D5  FA			cli				; Disable interrupts
020A:07D6  EB F8		jmp	short loc_8		; (07D0)

;; Выход из программы
020A:07D8	loc_10:
020A:07D8  C3			retn
			sub_7		endp
\end{lstlisting}

\clearpage

\section*{Схема алгоритма}

\img{220mm}{int8h_1.png}{Схема обработчика прерываний INT 8h}

\img{220mm}{int8h_2.png}{Схема обработчика прерываний INT 8h}

\img{220mm}{int8h_3.png}{Схема обработчика прерываний INT 8h}

\img{220mm}{sub_2.png}{Схема процедуры sub\_7}

\clearpage

%\section*{Вывод}

%Функции обработчика прерывания INT 8h в DOS:

%\begin{itemize}
%	\item Увеличивает текущее значение четырехбайтовой переменной, располагающейся в области данных BIOS по адресу 0000:046Ch. По этому адресу располагается счетчик тиков таймера. Если этот счетчик переполняется (после 24 часов с момента запуска таймера), в ячейку 0000:0470h заносится 1.
%	\item Контроль за работой двигателей моторчика дисковода. Если после последнего обращения к НГМД прошло более 2 секунд, обработчик прерывания выключает двигатель. Ячейка с адресом 0000:0440h содержит время, оставшееся до выключения двигателя. Это время постоянно уменьшается обработчиком прерывания таймера. Когда оно становится равно 0, двигатель НГМД отключается.
%	\item Вызов пользовательского прерывания 1Ch. Его стандартный обработчик состоит из одной команды IRET. Во время выполнения прерывания INT 1Ch все аппаратные прерывания запрещены.
%\end{itemize}